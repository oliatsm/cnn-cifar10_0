\chapter{Εισαγωγή} 
Η τεχνητή νοημοσύνη και η βαθιά μάθηση \en{(deep learning)} έχουν διεισδύσει σε πολυάριθμες πτυχές της καθημερινής ζωής. Οι τεχνολογίες αυτές χρησιμοποιούνται ολοένα και περισσότερο, από την ιατρική και την επιστημονική έρευνα έως τις εφαρμογές επεξεργασίας εικόνας και αναγνώρισης ομιλίας, καθώς και τους προσωπικούς βοηθούς στα κινητά μας τηλέφωνα. Αυτό είναι δυνατό λόγω της αυξανόμενης ισχύος υπολογιστικών συστημάτων και αρχιτεκτονικών, οι οποίες επιτρέπουν την επεξεργασία μεγάλου όγκου δεδομένων. 

Τα νευρωνικά δίκτυα, και ειδικότερα τα συνελικτικά νευρωνικά δίκτυα (\en{CNN}) έχουν αποδειχθεί εξαιρετικά αποτελεσματικά στην ανάλυση και ταξινόμηση εικόνων με μεγάλη ακρίβεια. Η αποτελεσματικότητά τους προέρχεται από την πολύπλοκη αρχιτεκτονική τους. Τα νευρωνικά δίκτυα συνιστούν μια μαθηματική μοντελοποίηση του τρόπου λειτουργίας του ανθρώπινου εγκεφάλου και της διαδικασίας μάθησης. Ουσιαστικά, εκτελούν μια σειρά μετασχηματισμών δεδομένων και εφαρμογές συναρτήσεων με εκπαιδεύσιμες παραμέτρους, διαδικασία που περιλαμβάνει πολλές επαναλήψεις και μαθηματικούς υπολογισμούς. 

Ένας σημαντικός παράγοντας που συνέβαλε στην εξάπλωση των νευρωνικών δικτύων ήταν η εξέλιξη των καρτών γραφικών (\en{GPU}) και η χρήση τους για προγραμματισμό γενικού σκοπού. Καθώς η κατασκευή γρηγορότερων επεξεργαστών έφτασε στα όριά της, λόγω ζητημάτων που αφορούν την κατανάλωση ενέργειας και την έκλυση θερμότητας από την αύξηση της ταχύτητας του ρολογιού, η αύξηση της επεξεργαστικής ισχύος σήμερα επιτυγχάνεται μέσω της χρήσης πολλαπλών πυρήνων. Οι κάρτες γραφικών διαθέτουν εκατοντάδες έως χιλιάδες επεξεργαστικούς πυρήνες που μπορούν να εκτελούν εντολές παράλληλα, καταναλώνοντας λιγότερη ενέργεια σε σχέση με τις κεντρικές μονάδες επεξεργασίας (\en{CPU}). Επιπλέον, η πρόσβαση σε αυτές είναι εύκολη καθώς πλέον υπάρχουν σε προσωπικούς υπολογιστές και ενσωματωμένα συστήματα. Έτσι, οι κάρτες γραφικών χρησιμοποιούνται τόσο στη φάση της εκπαίδευσης (\en{training}) όσο και στη φάση της εξαγωγής συμπερασμάτων (\en{inference}), δίνοντας αποτελέσματα σε πραγματικό χρόνο.


\section{Αντικείμενο της διπλωματικής}
\en{?? }Η παρούσα εργασία έχει ως σκοπό την αξιολόγηση της βελτιστοποίησης της ταχύτητας εκτέλεσης ενός συνελικτικού νευρωνικού δικτύου που ταξινομεί εικόνες, χρησιμοποιώντας τεχνικές παράλληλου προγραμματισμού. 
Υλοποίηση σειριακού κώδικα του συνελικτικού νευρωνικού δικτύου που κάνει ταξινόμηση εικόνων.
Η υλοποίηση έγινε σε γλώσσα \en{C}.
Σύνολο δεδομένων που ταξινομείτε είναι το \en{CIFAR-10}.
Χρήση παράλληλου προγραμματισμού για βελτιστοποίηση της ταχύτητας.
Αξιοποίηση της επεξεργαστικής ισχύος μιας κάρτας γραφικών.
Για προγραμματισμό σε κάρτα γραφικών χρησιμοποιήθηκε το προγραμματιστικό πρότυπο \en{OpenACC}, η παραλληλοποίηση γίνεται με εισαγωγή εντολών προς τον μεταγλωττιστή, διατηρώντας τον σειριακό κώδικα.
Αυτό προσφέρει φορητότητα, επιτρέποντας την εκτέλεση ακόμη και σε συστήματα χωρίς συμβατές κάρτες γραφικών.

\en{??} Πολλές είναι οι εφαρμογές όπου η γρήγορη εκτέλεση της εξαγωγής συμπερασμάτων είναι σημαντική. Η πιο χαρακτηριστική είναι στην αυτόνομη οδήγηση. Τα συνελικτικά νευρωνικά δίκτυα χρησιμοποιούνται για την αναγνώριση αντικειμένων όπως πεζοί, άλλα οχήματα, σήματα οδικής κυκλοφορίας και εντοπισμός εμποδίων, απαιτώντας γρήγορους χρόνους αντίδρασης για την αποφυγή ατυχημάτων. Αντίστοιχες απαιτήσεις υπάρχουν στη ρομποτική, όπου η πλοήγηση και η αλληλεπίδραση με το περιβάλλον πραγματοποιούνται αναλύοντας εικόνες ή βίντεο και λαμβάνοντας σήματα από διάφορους αισθητήρες.\cite{Ilas2020}

Στην ιατρική, τα συνελικτικά νευρωνικά δίκτυα χρησιμοποιούνται για την ανάλυση ιατρικών εικόνων, όπως ακτινογραφίες, μαγνητικές τομογραφίες και αξονικές τομογραφίες. Μπορούν να ανιχνεύσουν ανωμαλίες και ασθένειες όπως όγκους, κατάγματα και λοιμώξεις, βοηθώντας τους γιατρούς στην ακριβή διάγνωση. \cite{Ilas2020}, \cite{McKinney2020}

Η πρόβλεψη του καιρού και αλλαγές στο κλίμα είναι ένας ακόμη τομέας όπου η ταχύτητα εκτέλεσης των συνελικτικών νευρωνικών δικτύων έχει κάνει την έγκυρη ενημέρωση εφικτή. Τα μοντέλα που έχουν αναπτυχθεί παρέχουν ακριβείς πληροφορίες που σχετίζονται με την καλλιέργεια γης, τη ρύθμιση δρομολογίων στις μεταφορές και κρίσιμες πληροφορίες για την πρόληψη και διαχείριση φυσικών καταστροφών. \cite{Chen2023}

\section{Οργάνωση του τόμου}
\en{??} Η εργασία αυτή είναι οργανωμένη σε επτά κεφάλαια: Στο Κεφάλαιο 2
δίνεται το θεωρητικό υπόβαθρο των βασικών τεχνολογιών που
σχετίζονται με τη διπλωματική αυτή. Αρχικά περιγράφονται τα νευρωνικά δίκτυα, στη συνέχεια τα συνελικτικά νευρωνικά δύκτυα ... . Στο
Κεφάλαιο 3 αρχικά περιγράφεται η παράλληλη επεξεργασία
.... . Στο Κεφάλαιο 4 παρουσιάζεται η υλοποίηση του κώδικα και η παραλληλοποιηση του ....
Τέλος στο Κεφάλαιο 5 δίνεται η συνεισφορά αυτής της
διπλωματικής εργασίας, καθώς και μελλοντικές επεκτάσεις.